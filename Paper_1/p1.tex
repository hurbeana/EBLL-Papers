\documentclass[12pt, a4paper]{article}
\usepackage{helvet}
\usepackage{anyfontsize}
\usepackage[dvipsnames]{xcolor}
\usepackage{graphicx} 
\usepackage{tabularx}
\usepackage{fancyhdr}
\usepackage[ddmmyy]{datetime}
\usepackage{geometry}

%%%%%%%%%%%%%%%%%%%% PRÄAMBEL %%%%%%%%%%%%%%%%%%%% 
\renewcommand{\familydefault}{\sfdefault}
\renewcommand{\dateseparator}{.}
\geometry{a4paper, includehead, includefoot, top=2.5cm, bottom=2cm, left=2.5cm, right=2.5cm} 
\def\arraystretch{1.5}

\pagestyle{fancy}
\renewcommand{\headrulewidth}{0pt}
\renewcommand{\footrulewidth}{0pt}
\setlength{\headheight}{26pt}
\lhead{
	\raisebox{-.2\height}{
		\includegraphics{doc_images/EB.png}}
	\textcolor{gray}{\LARGE\ \&}
	\raisebox{-.3\height}{
		\includegraphics[scale=0.6]{doc_images/LL.png}}
	\hspace{.3cm}
	\textbf{\color{SpringGreen}2019S}
}
\rhead{\scalebox{.7}[1.0]{\large\textbf{Informatikdidaktik \& transferable skills}}}
\fancyfoot[L]{\footnotesize \bottomleftfooter}
\fancyfoot[C]{\footnotesize \today\ // \currenttime}
\fancyfoot[R]{\footnotesize \thepage}
%%%%%%%%%%%%%%%%%%%% PRÄAMBEL %%%%%%%%%%%%%%%%%%%% 

%%%%%%%%%%%%%%%%%%%% INFOS %%%%%%%%%%%%%%%%%%%% 
\newcommand{\authortext}{ENTER Author 1 \& Author 2}
\newcommand{\situation}{ENTER Situation}

\newcommand{\bottomleftfooter}{Situation 1 - ebll19s - de}
%%%%%%%%%%%%%%%%%%%% INFOS %%%%%%%%%%%%%%%%%%%% 

\begin{document}
\begin{tabular}{l l} 
Authors: & \textbf{\textcolor{blue}{\large\authortext}}\\ 
Situation & \textbf{\textcolor{blue}{\large\situation}}
\end{tabular}

\vspace{1em}

\centerline{
	\fcolorbox{white}{SpringGreen}{
		\fontsize{21}{12} \selectfont 
			Description of Situation / Triggers / Effects (table)}}

\vspace{1em}

\begin{table}[h!]
	\begin{tabularx}{\textwidth}{|p{0.3cm}|p{3.5cm}|X|p{0.3cm}|}
		\hline
		1 & Spielfeld-Nummer                       &  &  \\
		\hline
		2 & Ausgangspunkt                          &  &  \\
		\hline
		3 & Wirkungen                              &  &  \\
		\hline
		4 & Differenzierung \newline (falls nötig) &  &  \\
		\hline
		5 & Auslöser                               &  &  \\
		\hline
	\end{tabularx}
\end{table}
\end{document}