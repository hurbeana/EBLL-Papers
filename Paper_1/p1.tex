\documentclass[12pt, a4paper]{article}
\usepackage{lipsum}
\usepackage{helvet}
\usepackage{anyfontsize}
\usepackage[dvipsnames]{xcolor}
\usepackage{graphicx} 
\usepackage{tabularx}
\usepackage{enumitem}
\usepackage{fancyhdr}
\usepackage[ddmmyy]{datetime}
\usepackage{geometry}
\usepackage{cite}
\usepackage[ngerman]{babel}
\usepackage{bibgerm}


%%%%%%%%%%%%%%%%%%%% PRÄAMBEL %%%%%%%%%%%%%%%%%%%% 
\renewcommand{\familydefault}{\sfdefault}
\renewcommand{\dateseparator}{.}
\geometry{a4paper, includehead, includefoot, top=2.5cm, bottom=2cm, left=2.5cm, right=2.5cm} 
\def\arraystretch{1.5}

\pagestyle{fancy}
\renewcommand{\headrulewidth}{0pt}
\renewcommand{\footrulewidth}{0pt}
\setlength{\headheight}{26pt}
\lhead{
	\raisebox{-.2\height}{
		\includegraphics{doc_images/EB.png}}
	\textcolor{gray}{\LARGE\ \&}
	\raisebox{-.3\height}{
		\includegraphics[scale=0.6]{doc_images/LL.png}}
	\hspace{.3cm}
	\textbf{\color{SpringGreen}2019S}
}
\rhead{\scalebox{.7}[1.0]{\large\textbf{Informatikdidaktik \& transferable skills}}}
\fancyfoot[L]{\footnotesize \bottomleftfooter}
\fancyfoot[C]{\footnotesize \today\ // \currenttime}
\fancyfoot[R]{\footnotesize \thepage}
%%%%%%%%%%%%%%%%%%%% PRÄAMBEL %%%%%%%%%%%%%%%%%%%% 

%%%%%%%%%%%%%%%%%%%% INFOS %%%%%%%%%%%%%%%%%%%% 
\newcommand{\authortext}{Hurbean Alexander \& Ploder Bernhard}
\newcommand{\situation}{Fehlschläge}

\newcommand{\bottomleftfooter}{Situation 1 - ebll19s - de}
%%%%%%%%%%%%%%%%%%%% INFOS %%%%%%%%%%%%%%%%%%%% 

\begin{document}
\begin{tabular}{l l} 
Authors: & \textbf{\textcolor{blue}{\large\authortext}}\\ 
Situation: & \textbf{\textcolor{blue}{\large\situation}}
\end{tabular}

\vspace{1em}

\centerline{
	\fcolorbox{white}{SpringGreen}{
		\fontsize{21}{12} \selectfont 
			Description of Situation / Triggers / Effects (table)}}

\vspace{1em}

\begin{table}[h!]
	\begin{tabularx}{\textwidth}{|p{0.3cm}|p{3.5cm}|X|p{0.3cm}|}
		\hline
		1 & Spielfeld-Nummer                       & 35 – 39 Universität &  \\
		\hline
		2 & Ausgangspunkt                          & 
		Du hast seine Prüfung in einer Mathematik LVA schon zum zweiten Mal nicht geschafft.
		Nun bleiben dir 3 weitere Versuche, zwei davon sind kommissionell &  \\
		\hline
		3 & Wirkungen                              &
		Deine Reaktion darauf ist abhängig von deiner Persönlichkeit und vergangenen Erfahrungen. Nochmal Würfeln, abhängig vom Ergebnis:
		\begin{itemize}[noitemsep, topsep=0pt]
			\item {[1,2,3]} Du findest Prüfungsfach und Fachgebiet schwerer und unverständlicher (-1 Motivationspunkt).	Du hast erhöhte Prüfungsangst und glaubst, dass der Erfolg unwahrscheinlicher ist (-1 Motivationspunkt)
			\item {[4,5]} Vergangene Ergebnisse sind dir recht egal und du lernst für diese Prüfung ohne weitere Probleme (genauso wie für die ersten zwei Versuche)
			\item {[6]} Trotz Rückschlage bist du entschlossen und beschließt dich für den nächsten Prüfungsantritt möglichst gut vorzubereiten um deine Erfolgschancen zu maximieren, indem du mehr Zeit investierst und z.B. eine Nachhilfe aufsuchst (+1 Motivationspunkt)
		\end{itemize} &  \\
		\hline
		4 & Differenzierung \newline (falls nötig) & 
		\begin{itemize}[noitemsep, topsep=0pt]
			\item Persönlichkeit
			\item Soziales Umfeld
			\item Finanzielle Mittel
		\end{itemize}
		&  \\
		\hline
		5 & Auslöser                               & 
		\begin{itemize}[noitemsep, topsep=0pt]
			\item Negative/Positive vergangene Erfahrungen mit dem Prüfungsfach
			\item Sozialer Druck durch Kollegen/Freunden/Familie
			\item Andere Prüfungen die nicht erfolgreich waren
		\end{itemize}
		&  \\
		\hline
	\end{tabularx}
\end{table}
\newpage
\section*{Theoretische Begründung der behaupteten Wirkung/en}
	\lipsum[2-4]

\newpage
\section*{Empirische Belege für das Eintreten der behaupteten Wirkungen/en}
	\lipsum[3-5]
\bibliography{biblio}{}
\bibliographystyle{plain}
\end{document}