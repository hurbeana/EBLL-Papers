\documentclass[11pt, a4paper]{article}
\usepackage{lipsum}
\usepackage{blindtext}
\usepackage{helvet}
\usepackage{anyfontsize}
\usepackage[dvipsnames]{xcolor}
\usepackage{graphicx} 
\usepackage{tabularx}
\usepackage{enumitem}
\usepackage{fancyhdr}
\usepackage[ddmmyy]{datetime}
\usepackage{geometry}
\usepackage{cite}
\usepackage[ngerman]{babel}
\usepackage{bibgerm}
\usepackage{ragged2e}
\usepackage{hyperref}
\usepackage[german]{cleveref}


%%%%%%%%%%%%%%%%%%%% PRÄAMBEL %%%%%%%%%%%%%%%%%%%% 
%Dates
\renewcommand{\familydefault}{\sfdefault}
\renewcommand{\dateseparator}{.}
\setlength{\parindent}{1.5em}
\setlength{\parskip}{0.5em}

%Papersize
\geometry{a4paper, includehead, includefoot, top=2.5cm, bottom=2cm, left=2.5cm, right=2.5cm} 

%Tablestuff
\def\arraystretch{1.5}
\newcolumntype{L}{>{\RaggedRight} X}

%Headers
\pagestyle{fancy}
\renewcommand{\headrulewidth}{0pt}
\renewcommand{\footrulewidth}{0pt}
\setlength{\headheight}{26pt}
\lhead{
	\raisebox{-.2\height}{
		\includegraphics{doc_images/EB.png}}
	\textcolor{gray}{\LARGE\ \&}
	\raisebox{-.3\height}{
		\includegraphics[scale=0.6]{doc_images/LL.png}}
	\hspace{.3cm}
	\textbf{\color{SpringGreen}2019S}
}
\rhead{\scalebox{.7}[1.0]{\large\textbf{Informatikdidaktik \& transferable skills}}}
\fancyfoot[L]{\footnotesize \bottomleftfooter}
\fancyfoot[C]{\footnotesize \today\ // \currenttime}
\fancyfoot[R]{\footnotesize \thepage}
%%%%%%%%%%%%%%%%%%%% PRÄAMBEL %%%%%%%%%%%%%%%%%%%% 

%%%%%%%%%%%%%%%%%%%% INFOS %%%%%%%%%%%%%%%%%%%% 
\newcommand{\authortext}{Hurbean Alexander \& Ploder Bernhard}
\newcommand{\situation}{Game-Based Learning}

\newcommand{\bottomleftfooter}{Situation 2 - ebll19s - de}
%%%%%%%%%%%%%%%%%%%% INFOS %%%%%%%%%%%%%%%%%%%% 

\begin{document}
\begin{tabular}{l l} 
Authors: & \textbf{\textcolor{blue}{\large\authortext}}\\ 
Situation: & \textbf{\textcolor{blue}{\large\situation}}
\end{tabular}

\vspace{1em}

\centerline{
	\fcolorbox{white}{SpringGreen}{
		\fontsize{21}{12} \selectfont 
			Description of Situation / Triggers / Effects (table)}}

\vspace{1em}

\begin{table}[h!]
	\begin{tabularx}{\textwidth}{|p{0.3cm}|p{3.5cm}|L|}
		\hline
		1 & Spielfeld-Nummer                       & 04 – 11 Volksschule/Unterstufe \\
		\hline
		2 & Ausgangspunkt                          & 
		Auf deiner Schule wird ein \textit{Game-Based Learning} Ansatz, in verschiedenen Fächern, ausprobiert.\\
		\hline
		3 & Wirkungen                              &
		\begin{itemize}[topsep=0pt]
			\item {[1,2,3,4,5]} Durch das neu eingeführte \textit{Game-Based Learning} findest du den Unterricht wesentlich interessanter und kannst dich länger konzentrieren. Die Folge davon sind Spaß am Lernen und bessere Noten.\newline\texttt{($+5$ Motivationspunkte)}
			\item {[6]} Das Game-Based Learning hatte keinen guten Effekt auf dich und du hast nicht mehr gelernt als sonst.\newline\texttt{($\pm 0$ Motivationspunkte)}
		\end{itemize} \\
		\hline
		4 & Differenzierung \newline (falls nötig) & \\
		\hline
		5 & Auslöser                               & 
			Du befindest dich in einem Schuljahr in dem Unterrichtsfächer stattfinden die das neue, alternative Lernverfahren einsetzen. \\
		\hline
	\end{tabularx}
\end{table}
\newpage

\section*{Theoretische Begründung der behaupteten Wirkung/en}
\blindtext

\newpage
\section*{Empirische Belege für das Eintreten der behaupteten Wirkungen/en}

\blindtext
\newpage

\bibliography{biblio}{}
\bibliographystyle{alpha}

\end{document}